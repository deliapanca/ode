\documentclass{beamer}
%svgnames
%
\usepackage[all]{xypic}
\usepackage{graphicx}
\usepackage{amsfonts}
\usepackage{amssymb}
\usepackage{amsmath}
\mode<presentation> {
\usetheme{CambridgeUS}}
%\setbeamercovered{transparent}}
\setbeamertemplate{items}[default]

%\useinnertheme{rounded}
\title[Conference]{\'Ciri\'c type fixed point theorems}
\author[A. Petru\c sel]{Adrian Petru\c sel}
\institute[BBU Cluj-Napoca]{Babe\c s-Bolyai University Cluj-Napoca\\
Faculty of Mathematics and Computer Science}
\date{January, 2014}



\AtBeginSection[]{
%\frametitle{Planul prezent\u arii}
\tableofcontents[current]}

\AtBeginSubsection[] {

 \begin{frame}<beamer>
   %\frametitle{Planul prezent\u arii}
    \tableofcontents[currentsection,currentsubsection]
  \end{frame}}



\begin{document}
\begin{frame}
\titlepage
\end{frame}

\section{Introduction}
\textrm{}
%%%%%%%%%%%%%%%%%%%%%%%%%%%%%%%%%%%%%%%%%%%%%%%%%%%%%%%%%%%%%%%
\subsection{Abstract}
\begin{frame}

The purpose of this talk is to present
some fixed point and strict results for (single-valued and multi-valued) 
generalized contractions of \'Ciri\'c type.\pause



\medskip


Connections with other results given in: 

\end{frame}

\begin{frame}


L.B. \'Ciri\'c: \textit{Generalized contraction and fixed point theorems}, 
Publ. Inst. Math., 12 (1971), 19-26.\pause

\medskip

L.B. \'Ciri\'c: {\it Fixed points for generalized multi-valued contractions},
Math. Vesnik, 9(24) (1972), 265-272.\pause

\medskip

L.B. \'Ciri\'c: \textit{A generalization of Banach's contraction principle}, 
Proc. Amer. Math. Soc., 45 (1974), 267-273.\pause

\medskip


L.B. \'Ciri\'c: {\it Contractive type non-self mappings on metric spaces of
hyperbolic type}, J. Math. Anal. Appl., 317 (2006), 28-42.\pause

\medskip

G.E. Hardy and T.D. Rogers: {\it A generalization of a fixed point theorem of Reich}, Canad. Math. Bull., 16(1973), 201-206.

\medskip

S. Reich: \textit{Fixed point of contractive functions}, Boll. Un. Mat. Ital.,
5 (1972), 26-42.\pause




\end{frame}

\begin{frame}

I.A. Rus: {\it Generalized Contractions and Applications}, Transilvania Press, 2001.\pause

\medskip

I. Beg, A.R. Butt and S. Radojevic: \textit{The contraction
principle for set-valued mappings on a metric space with a graph},
Computers Math. Appl., 60(2010), 1214-1219.\pause

\medskip

A. Petru\c sel, G. Petru\c sel: Multivalued Picard operator, J. Nonlinear Convex Anal., 2012
\pause


T. Dinevari and M. Frigon: \textit{Fixed point results for
multivalued contractions on a metric space with a graph}, J. Math. Anal.
Appl., 405(2013), 2, 507-517.\pause

\medskip


C. Chifu, G. Petru\c sel and M. Bota,
{\it Fixed points and strict fixed points for multivalued contractions of Reich type on metric 
spaces endowed with a graph}, Fixed Point Theory  Appl. 2013, 2013:203, 
doi:10.1186/1687-1812-2013-203.\pause

\end{frame}

\section{Some known results}

\subsection{The single-valued case}


\subsection{The multi-valued case}

\section{Fixed point theorems in metric spaces endowed with a partial order}

\subsection{Basic concepts}

\begin{frame}
 \vskip3mm \pause
  \begin{itemize}
  \item
{\bf Definition.} Let $X$ be a nonempty set. Then, by
definition $(X, \to, \le)$ is an ordered L-space if and only if:

\quad (i) $(X,\to)$ is an L-space;

\quad (ii) $(X,\le)$ is a partially ordered set;

\quad (iii) $(x_n)_{n\in \mathbb{N}}\rightarrow x$, $(y_n)_{n\in
\mathbb{N}}\rightarrow y$ and $x_n\le y_n$, for each $n\in
\mathbb{N} \mbox{ } \Rightarrow x\le y$.

\medskip

\item If $(X,d)$ is a metric space, then the triple
$(X,d,\le)$ will be called an ordered metric space.

\end{itemize}
\end{frame}

\begin{frame}
\begin{itemize}
\item 
Let $(X,\le)$ be a partially ordered set. 

Denote $$X_{\le}:=
\{(x,y)\in X\times X| x\le y \mbox{ or } y\le x \}.$$ \pause

\item In the same setting, consider $f:X\to X$. Then:

$(LF)_f:=\{x\in X| x\le f(x) \}$ is the lower fixed point set of $f$, 

$(UF)_f:=\{x\in X| x\ge f(x) \}$ is the upper fixed point
set of f. \pause

\bigskip

\item If $f:X\to X$ and $g:Y\to Y$, then the cartesian
product of $f$ and $g$ is denoted by $f\times g$ and it is defined
in the following way: $$f\times g:X\times Y\to X\times Y, (f\times
g)(x,y):= (f(x),g(y)).$$
\end{itemize}
\end{frame}

\begin{frame}

{\bf Definition.} (I.A. Rus) Let $(X,\rightarrow)$
be an L-space.\\ An operator $f:X\to X$ is, by definition, a Picard
operator if:

(i) $F_f=\{x^*\}$;

(ii) $(f^{n}(x))_{n\in\mathbb{N}}\to x^*$ as $n\to\infty ,\ \mbox{
for all }  x\in X$.

\end{frame}


\begin{frame}
  %\frametitle{Obiectivele tezei}
\begin{exampleblock}{} 

{\bf Theorem.} (Ran and Reurings-2004)  {\it Let $X$ be a
partially ordered set such that every pair $x,y\in X$ has a lower
and an upper bound. Let $d$ be a metric on $X$ such that the metric
space $(X,d)$ is complete. Let $f:X\to X$ be a continuous and
monotone (i. e., either decreasing or increasing) operator. Suppose
that the following two assertions hold:

\quad 1) there exists $a\in ]0,1[$ such that $d(f(x),f(y))\le
a\cdot d(x,y)$, for each $x,y\in X$ with $x\le y$

\quad 2) $(LF)_f\cup (UF)_f\neq\emptyset$.

Then $f$ is a Picard operator.} \end{exampleblock}\end{frame}

\begin{frame}

{\bf Theorem.} (Nieto and Rodr\'{\i}guez-L\'{o}pez, 2005)
{\it Let $X$ be a partially ordered set such that every pair $x,y\in
X$ has a lower or an upper bound. Let $d$ be a metric on $X$ such
that the metric space $(X,d)$ is complete. Let $f:X\to X$ be an
increasing operator. Suppose that the following two assertions hold:

\quad 1) there exists $a\in ]0,1[$ such that $d(f(x),f(y))\le
a\cdot d(x,y)$, for each $x,y\in X$ with $x\le y$;

\quad 2) there exists $x_0\in X$ such that $x_0\le f(x_0)$;

\quad 3) if an increasing sequence $(x_n)$ converges to $x$ in $X$,
then $x_n\le x$ for all $n\in \mathbb{N}$.

Then $f$ is a Picard operator.}

\end{frame}


\subsection{Some abstract results}
\begin{frame}


\begin{block}{\bf Lemma.}(A. Petru\c sel and Rus, 2006)

{\it Let $(X, \to)$ be an L-space and $U$ a
symmetric subset of $X\times X$ such that $\Delta(X)\subset U$.
Let $f:X\to X$ be an operator. Suppose that:

\quad (i) for each $x,y\in X$ with $(x,y)\notin U$ there exists
$z\in X$ such that $(x,z)\in U$ and $(y,z)\in U$;

\quad (ii) there exist $x_0,x^*\in X$ such that $x_0\in A_f(x^*)$;

\quad (iii) $(x,y)\in U$ and $x\in A_f(x^*)$ implies $y\in
A_f(x^*)$.

Then $A_f(x^*)=X$.

Moreover, if $f$ is orbitally continuous, then $f$ is a Picard operator.}
\end{block} \vskip 3mm

where $$A_f(x^*):=\{x\in X \ : \ (f^{n}(x))_{n\in\mathbb{N}}\to x^* \mbox{ as } n\to\infty \}.$$
\end{frame}

\begin{frame}

A natural consequence of the above result follows by choosing
$U:=X_{\le}$.

\begin{block}{\bf Lemma.}(A. Petru\c sel and Rus, 2006) 

{\it Let $(X, \to, \le)$ be an ordered L-space
and  $f:X\to X$ be an operator. Suppose that:

\quad (i) for each $x,y\in X$ with $(x,y)\notin X_{\le}$ there
exists $z\in X$ such that $(x,z)\in X_{\le}$ and $(y,z)\in
X_{\le}$;

\quad (ii) there exist $x_0,x^*\in X$ such that $x_0\in A_f(x^*)$;

\quad (iii) $(x,y)\in X_{\le}$ and $x\in A_f(x^*)$ implies $y\in
A_f(x^*)$;

\quad $(iv)$ $f$ is orbitally continuous


Then $f$ is a Picard opeartor.}\end{block} 

\end{frame}

\subsection{Fixed point theorems for $\varphi$-contractions}


\begin{frame}
\begin{block}
{\bf Theorem.}(O'Regan and A. Petru\c sel, 2008)

{\it Let $(X,d,\le)$ be an ordered metric space
and $f:X\to X$ be an operator. We suppose that:

\quad (i) For each $x,y\in X$ with $(x,y)\notin X_{\le}$ there
exists $c(x,y)\in X$ such that $(x,c(x,y))\in X_{\le}$ and
$(y,c(x,y))\in X_{\le}$;

\quad (ii) $f:(X,\le)\to (X,\le)$ is  increasing;

\quad (iii) there exists $x_0\in X$ such that $x_0\le f(x_0)$;

\quad $(iv)_{a}$ \quad $f$ is orbitally continuous

\quad or

\quad $(iv)_b$ \quad if an increasing  sequence $(x_n)$ converges
to $x$ in $X$, then $x_n\le x$  for all $n\in \mathbb{N}$;


\quad (v) there exists a comparison function $\varphi:\mathbb{R}_+\to \mathbb{R}_+$
such that $d(f(x),f(y))\le \varphi(d(x,y))$, for each $(x,y)\in
X_{\le}$;

\quad (vi) the metric $d$ is complete.

Then $f$ is a Picard operator.}
\end{block} 
\end{frame}



\section{Fixed point theorems in metric spaces endowed with a graph}

\subsection{Basic concepts}

\begin{frame}


Let $(X,d)$ be a metric space and $\Delta$  the
diagonal of $X\times X$.\pause

\medskip

Let  $G$ be a directed graph such that one can identify $G$ with the pair $(V(G),E(G))$, where the set $V(G)$ of its vertices coincides
with $X$  and the set $E(G)$ of the edges of the graph contains $\Delta$.\pause 

\medskip

If $x$ and $y$ are vertices of $G$, then a path in $G$ from $x$ to $y$ of length $k\in \mathbb{N}$ is a finite sequence 
$(x_n)_{n\in \{0,1,2,\cdots, k \}}$ of vertices such that
$$x_0 = x, x_k = y \mbox{  and } (x_{i-1}, x_i)\in E(G), \mbox{ for } i\in \{1,2,\cdots, k \}.$$\pause

\medskip

A graph $G$ is connected if
there is a path between any two vertices and it is weakly connected if $\tilde{G}$ is connected, where
$\tilde{G}$ denotes the undirected graph obtained from $G$ by ignoring the direction of edges.

\end{frame}

\begin{frame}
\begin{block} {\bf Definition. (J. Jachymski-2008)}
An operator $f:X\to X$ is called a Banach $G$-contraction  if and only if:\pause

\quad (a) $f$ is edge preserving, i.e., for each $x,y\in X$ with $(x,y)\in E(G) \mbox{ we have } (f(x),f(y))\in E(G)$;\pause

\quad (b) there exists $\alpha\in ]0,1[$ such that for each $x,y\in X$ the following implication holds:
$$(x,y)\in E(G) \mbox{ implies } d\left(f(x),f(y)\right)\le \alpha d(x,y).$$
\end{block} \vskip2mm\pause
Examples:\\
1) Any Banach contraction is a $G_0$-contraction, where the graph $G_0$
is defined by $E(G_0):= X\times X$.\vskip1mm\pause
2) Let $\le$ be a partial order in X. Define the graph $G_1$ by
$$E(G_1):= \{(x, y)\in X\times X \ : \ x\le y \}.$$\pause
3) Let $\le$ be a partial order in X. Define the graph $G_2$ by
$$E(G_2):= \{(x, y)\in X\times X \ : \ x\le y \mbox{ or } y\le x \}.$$

\end{frame}

\subsection{The single-valued case}

\begin{frame}
\begin{block} {\bf Theorem. (J. Jachymski-2008)}
{\it Let $(X,d)$ be a complete metric space and let $G$ be a
directed graph $G$ such that  the triple $(X,d,G)$ has property $(P)$:
\[ (P)  \left. \begin{array}{lll}
	  \mbox{for any sequence } (x_{n})_{n\in \mathbb{N}}\subset X, \mbox{ if } x_n\to x \mbox{ as } n\to +\infty \mbox{ and }\\ (x_n,x_{n+1})\in E(G),
      \mbox{for each } n\in \mathbb{N}, \mbox{ then there exists a subsequence }\\ (x_{k_n})_{n\in \mathbb{N}} \mbox{ of } (x_n)_{n\in \mathbb{N}} \mbox{ such that }
      (x_{k_n},x)\in E(G), \mbox{ for each } n\in \mathbb{N}.\end{array}\right.\]
Let $f:X\to X$ be a $G$-contraction. Then the following statments hold:

\quad 1) $F_f\neq\emptyset$ if and only if $X_f\neq \emptyset$, $$\mbox{ where } X_f:=\{x\in X \ : \ (x,f(x))\in E(G) \};$$

\quad 2)   if $X_f\neq \emptyset$ and $G$ is weakly connected, then $f$ is a Picard operator.}
\end{block} 
\end{frame}



\subsection{The multi-valued case}

\begin{frame}
\begin{block} {\bf Theorem.}(Nicolae-O'Regan-A. Petru\c sel, 2011)

{\it Let $(X,d)$ be a complete metric space and $G$ be a directed graph such that the triple $(X,d,G)$ satisfy property $(P)$.
Let $T : X \to P_{cl}(X)$ be a multi-valued operator. Suppose the following assertions hold: 

\quad (i) there exists $\alpha\in (0,1)$ such that $$H(T(x),T(y)) \le \alpha d(x,y) \mbox{ for all } (x,y) \in E(G).$$

\quad (ii) for each $(x,y)\in E(G)$, each $u\in T(x)$ and $v\in T(y)$ satisfying the condition
$d(u,v)\le a d(x,y)$, for some $a\in (0,1)$, we have $(u,v)\in E(G)$;

Then $F_T\neq\emptyset$ if and only if $X_T\neq\emptyset$, where 
$$X_T:=\{x\in X ; \mbox{ there exists } y\in  T(x) \mbox{ such that } (x,y)\in E(G) \},$$.}
\end{block} 
\end{frame}

\section{Main works}
\begin{frame}

This part is based on:\pause \vskip1mm

\begin{block}{A.C.M. Ran, M.C. Reurings:} A fixed point theorem in partially
ordered sets and some applications to matrix equations, {\it Proc.
Amer. Math. Soc.}  132(2004) 1435-1443.
\end{block}\pause

\begin{block}{J.J. Nieto, R. Rodr\'{\i}guez-L\'{o}pez:}  Contractive mapping theorems in
partially ordered sets and applications to ordinary differential
equations, {\it Order} 22(2005)  223-239.
\end{block}\pause 


\begin{block}{A. Petru\c sel, I.A. Rus:}
Fixed point theorems in ordered
$L$-spaces, {\it Proc. Amer. Math. Soc.} 134(2006) 411-418.
\end{block}\pause

\begin{block}{D. O'Regan, A. Petru\c sel:} 
Fixed point theorems in ordered metric spaces,
{\it J. Math. Anal. Appl.} 341(2008) 1241-1252.
\end{block}
\end{frame}


\begin{frame}

\begin{block}{J. Jachymski:}
The contraction principle for mappings on a metric space with a graph,
{\it Proc. Amer. Math. Soc.} 136(2008) 1359-1373.
\end{block}\pause 


\begin{block}{A. Nicolae, D. O'Regan, A. Petru\c sel:}
Fixed point theorems for singlevalued and multivalued generalized contractions in metric spaces
endowed with a graph, J. Georgian Math. Soc., 18(2011), 307�327.
\end{block}
\end{frame}



\end{document}
